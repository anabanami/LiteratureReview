
%----------------------------------------------------------------------------------------
%   PACKAGES AND OTHER DOCUMENT CONFIGURATIONS
%----------------------------------------------------------------------------------------

% latexmk -pvc -pdf
\documentclass[12pt, a4paper]{report}
\usepackage[margin=2.5cm]{geometry}
\usepackage{titlesec}
\usepackage{amsmath,amsthm,amsfonts,amssymb}
\usepackage{mathtools}
\usepackage{braket}
\usepackage{dsfont}
\usepackage[font=scriptsize,labelfont=bf]{caption}
\usepackage[english]{babel}
\usepackage{graphicx}
\newenvironment{Figure}
    {\par\medskip\noindent\minipage{\linewidth}}
    {\endminipage\par\medskip}
\usepackage{blindtext}

% Colours:
\usepackage[table]{xcolor}
\definecolor{purple}{RGB}{117,77,226}

\usepackage[bookmarksopen,
  pagebackref,
  pdfpagelayout=TwoPageRight,
  colorlinks=true,
  urlcolor=purple,
  citecolor=purple,
  filecolor=purple,
  linkcolor=purple,
  urlcolor=purple,
  citecolor=purple,
  filecolor=purple,
  linkcolor=purple,
  linktocpage=true
  ]
{hyperref}

\titleformat{\chapter}[display]
  {\normalfont\bfseries}{}{0pt}{\Huge}

\newcommand\II{\(\mathcal{I}\)}
\newcommand\PT{\(\mathcal{PT}\)}
\newcommand\PP{\(\mathcal{P}\)}
\newcommand\TT{\(\mathcal{T}\)}
\newcommand\CPT{\(\mathcal{CPT}\)}
\newcommand\CC{\(\mathcal{C}\)}

%----------------------------------------------------------------------------------------
%   TITLE PAGE
%----------------------------------------------------------------------------------------

\begin{document}
\begin{titlepage}
\begin{center}


\vspace{1cm}
\textsc{Honours Literature review}
\vspace{2.5cm}

{\Huge \( \mathcal{PT} \)-symmetric quantum mechanics}
\vspace{3cm}

{\LARGE Ana Fabela Hinojosa \footnote{acfab1@student.monash.edu.au}} \\
\vspace{0.4cm}
{\Large Supervisors:\\ Dr. Jesper Levinsen \\ Prof. Meera Parish \\}
\textsc{School of Physics \& Astronomy} \\
\vspace{3cm}
\includegraphics[scale=0.2]{logo.jpg} \\ % University logo
\vspace{3cm}
{\LARGE August 2022}\\
\vspace{0.5cm}
\end{center}
\end{titlepage}

%----------------------------------------------------------------------------------------
%   QUOTATION PAGE
%----------------------------------------------------------------------------------------
% \vspace*{0.2\textheight}

% \noindent{``How do you know I’m mad?'' said Alice.

% ``You must be,'' said the Cat, ``or you wouldn't have come here.''}\bigbreak

% \hfill  Lewis Carroll, Alice in Wonderland 

% \vspace{20cm}
%%----------------------------------------------------------------------------------------
%%   LIST OF CONTENTS/FIGURES/TABLES PAGES
%%----------------------------------------------------------------------------------------

\tableofcontents % Prints the main table of contents

% \vspace{20cm}

%\listoffigures % Prints the list of figures

%\vspace{20cm}

%% \listoftables % Prints the list of tables

% %----------------------------------------------------------------------------------------
% %  CONTENTS
% %----------------------------------------------------------------------------------------

\begin{abstract}\label{Abstract}
\end{abstract}
%Quantum mechanical operators must satisfy a set of properties which deem them suitable as observable dynamical variables in nature. Observers use operators to measure some qualities of a system (such as for example a system's total energy) and come to a real valued finite answer.\\

\chapter{Introduction}\label{Introduction}
This work reflects on a possible extension to the canonical formalism of quantum mechanics. The main goal of this extension is to allow us to access a much larger class of interesting Hamiltonians which are non-Hermitian but nevertheless physical.This extension has the potential to be of great importance to the advancement of new and physically significant theories.\\
In ``The principles of quantum mechanics", Paul Dirac advises us that``it is important to remember that science is concerned only with \textit{observable} things and that we can observe an object only by letting it interact with some outside influence"\cite{POQM}. This means that in order to study valid physical systems, these systems must satisfy the fundamental postulates of quantum mechanical theory. Nearly all these postulates are based in physical properties. For example, one postulate establishes that time evolution of a quantum system must be \textit{unitary} (i.e. probability conserving). Another requires that the energy spectrum of the system is bounded below so a lowest energy state can be measured. In quantum mechanics, the Hamiltonian operator $(\hat{H})$ encapsulates the total energy of a system. As explained already, a system's energy spectrum is required to be bounded and real in order to be measurable. According to the fundamental postulates of the standard theory, if the operator $\hat{H}$ satisfies the mathematical property known as Hermiticity --these operators are also known as self-adjoint in mathematics--, then $\hat{H}$ will be an adequate physical observable. Interestingly, the Hermiticity postulate stands out from others in the conventional theory of quantum mechanics because it is mathematical rather than physical in its character\cite{MakingSense}.
An operator $\hat{O}$ that is Hermitian has the property that its effect on the vectors of the Hilbert space in which $\hat{O}$ is defined is independent of the order in which $\hat{O}$ acts on said vectors\cite{Jones-Smith}
\begin{equation}\label{eq:1}
\hat{O}\ket{\psi} = \bra{\psi}\hat{O}^{\dagger}.
\end{equation}
Despite of the correctness of Hermiticity, some believe that perhaps we have ended up with an overly restrictive quantum theory.  
The aim of this review is to summarize the present developments of a more physical alternative to Hermiticity. This alternative postulate is referred to as space–time reflection symmetry (\PT\:symmetry)\cite{MustaHbeHermitian}. 

\section{\texorpdfstring{$\mathcal{PT}$}\:-symmetry}\label{PT}
In the late nineties, Carl Bender et al presented a \PT-symmetric theory of quantum mechanics. Their aim was to explain a conjecture on the reality and positiveness of the spectrum of a non-Hermitian Hamiltonian proposed by Bessis\cite{Bender1998}. \PT-symmetric theory can be viewed as an analytical continuation of the conventional theory from real into the complex phase space\cite{PT-symmetricQM}.

An important question that must be answered, is whether a \PT-symmetric Hamiltonian defines a valid physical theory of quantum mechanics. By a physical theory we mean that the following conditions must be satisfied; The energy spectra of a system described by $\hat{H}$ must be real and bounded below. Another condition is related to the probabilistic interpretation of the norm of a state, a norm must be always positive to be a valid probability. Finally time evolution under the theory must be unitary. This means that as a state vector evolves in time the state's probability does not leak away\cite{MustaHbeHermitian}\cite{MakingSense}.

\subsection{The \texorpdfstring{$\mathcal{PT}$}\: operator}
The \PT\:operator is the anti-linear operator composed of the linear parity operator (\PP), which performs spatial reflection, and the anti-linear time-reversal operator (\TT). These operators act on position and momentum operators in the following form
\begin{equation}\label{eq:2}
\begin{split}
\mathcal{P}:& \quad\hat{x} \rightarrow -\hat{x},\quad \hat{p} \rightarrow -\hat{p},\\
\mathcal{T}:& \quad\hat{x} \rightarrow \hat{x},\quad\quad \hat{p} \rightarrow -\hat{p},\quad \mathrm{i} \rightarrow -\mathrm{i}.
\end{split}
\end{equation}
Some Hamiltonians may not be symmetric under \PP\:or \TT\:separately, but Hamiltonians that remain invariant under the influence of the \PT\:operator are labelled as \PT-symmetric. A Hamiltonian $\hat{H}$ will possess unbroken \PT\:symmetry if it's eigenstates are simultaneously eigenstates of the \PT\: operator, or in other words, if $\hat{H}$ and the \PT\:operator commute. If the \PT\:operator and $\hat{H}$ do not commute we say that the Hamiltonian's \PT-symmetry is broken\cite{MakingSense}\cite{ComplexExtension}\cite{MustaHbeHermitian}. If the symmetry is unbroken, then the eigenspectrum of $\hat{H}$ is fully real and bounded below --This is also known as exact \PT-symmetry--. The effectiveness of \PT\:symmetry as a tool to investigate the spectra of some non-Hermitian Hamiltonians has been proved rigorously in various works, such as Dorey et al\cite{Dorey_2001}, Bender and Boettcher\cite{Bender1998}, Brody\cite{Brody_2016}, Bender and Mannheim \cite{Bender_2010}, Bender et al\cite{PT-symmetricQM}, Mostafazadeh\cite{Mostafazadeh}\cite{Mostafazadeh2} amongst several others. 

\subsection{The \texorpdfstring{$\mathcal{CPT}$}\:\:inner product}\label{CPT}
To be able to describe precisely the nature of \PT-symmetric quantum mechanics, we must delve briefly into the inner-product under which our theory satisfies the postulates of conventional quantum mechanics. It is important to note that \PT-symmetric quantum mechanics is a kind of `bootstrap' theory\cite{MakingSense}, since infinitely many inner-products exist for a given vector space, we can construct an inner product whose associated norm is positive definite by design. This inner-product is in general dependent on the characteristics of the Hamiltonian in question and it guarantees that the underlying dynamics of any \PT-symmetric Hamiltonian satisfies unitarity\cite{MustaHbeHermitian}.
Firstly, it is necessary to solve for the eigenstates of the Hamiltonian before knowing the Hilbert space and consequentially the associated inner product.
To guarantee a positive norm for our theory, we will construct a new linear operator \CC\:that commutes with both $\hat{H}$ and \PT. We use the symbol \CC\: to represent this symmetry because it's properties are similar to those of the charge conjugation operator in particle physics\cite{MakingSense}.

\subsubsection{The $\mathcal{C}$ operator}\label{CC}
When the \PT-symmetry of $\hat{H}$ is exact, then $\hat{H}$ and \PT\:commute. This statement is equivalent to saying that the eigenfunctions $\phi_n(x)$ of $\hat{H}$ are simultaneously eigenstates of \PT\cite{Bender_2004}.
\begin{equation}\label{eq:3}
\mathcal{PT}\phi_n(x) = \lambda_n \phi_n(x),
\end{equation}
Where $\lambda_n$ is a pure phase. Without loss of generality, for each $n$\:the phase can be absorbed into $\phi_n(x)$ and this makes the eigenvalue of the \PT operator unity\cite{Bender_2004}: 
\begin{equation}\label{eq:4}
\mathcal{PT}\phi_n(x) = \phi_{n}^{*}(-x) = \phi_n(x).
\end{equation}
%%%%%%%%%%%%%%%%%%%%%%%%%%%%%%%%%%%%%%%%%%%%%%%%%%%%%%EDITS UP TO HERE DONE

There is strong numerical evidence of the completeness of the eigenfunctions $\phi_n(x)$\cite{ComplexExtension}\cite{Bender_2004}\cite{Brody_2013}. In the coordinate basis, the completeness statement reads:
\begin{equation}\label{eq:5}
\sum_{n}(-1)^{n}\phi_n(x)\phi_n(y) = \delta(x-y),\quad x, y \in \mathbb{R}
\end{equation}
the unconventional $(-1)^n$ factor in \ref{eq:5} can be explained if we define the \PT\:inner product as
\begin{align}\label{eq:6}
\left ( f, g \right )  & = \int dx \left [ \mathcal{PT} f(x) \right ] g(x)\nonumber\\
                       & = \int dx f^{*}(-x) g(x).
\end{align}
where the integral above follows a path in the complex plane. Under this definition the eigenstate norms alternate in sign depending on the value $n$. This means that the metric associated with the \PT\: inner product is indefinite\cite{Bender_2004}\cite{Critique}.
In quantum theory, the norm of states is interpreted as a probability and this means that the indefinite metric described above presents a serious problem for the validity of \PT-symmetric quantum theory. The solution to this problem lies in finding an interpretation for
the negative valued norms\cite{PT-symmetricQM}. The general claim presented in the literature is that for any theory with unbroken \PT-symmetry there
exists a symmetry of the Hamiltonian that describes the negative and positive norm states. To describe this symmetry of $\hat{H}$ it is necessary to construct a linear operator denoted by \CC\cite{MustaHbeHermitian}\cite{ComplexExtension}\cite{Bender_2004}. When represented in position space \CC\: is a sum over the energy eigenstates of $\hat{H}$:
\begin{equation}\label{eq:7}
\mathcal{C} = \sum_n \phi_n(x)\phi_n(y)
\end{equation}
From this definition, and the relation $(\phi_m(y), \phi_n(y)) = \int dy\:\phi_m(y)\phi_n(y)$ we can verify that the eigenvalues of \CC\: are $\pm 1$
\begin{align}\label{eq:7}
\mathcal{C} \phi_n(x) & = \int dy\:\mathcal{C}\:\phi_n(y)\nonumber \\
& = \sum_{m}\phi_m(x)\int dy\:\phi_m(y) \phi_n(y)\nonumber \\
& = (-1)^n \phi_n(x)
\end{align}
The \PT\:norms signatures can therefore be interpreted as the ``charge'' of the states, while \CC\: is the operator used to measure this charge\cite{Bender_2004}.

The \CC\: operator commutes with the \PT\: operator, but it does not commute with the parity operator \PP. Notice that \CC\:and \PP\: operators are square roots of $\delta(x-y)$ the unity operator\cite{ComplexExtension}
\begin{equation}\label{eq:8}
\mathcal{P}^2 = \mathcal{C}^2 = \mathds{1}, 
\end{equation}
where $\mathcal{P} \neq \mathcal{C}$, since \PP\: is real and \CC\: is complex valued\cite{MustaHbeHermitian}\cite{Bender_2004}.

Using the newly constructed \CC\: operator, we can redesign the \PT\: inner product to suit the conventional probabilistic interpretation of the vector norms in quantum mechanics

\begin{equation}\label{eq:9}
\left( f, g \right ) = \int dx \left [ \mathcal{CPT} f(x) \right ] g(x).
\end{equation}


\chapter{Equivalent Hamiltonians with distinct symmetries}\label{Equiv}

In parallel to the work of Bender is the research of Mostafazadeh who introduced the notion of pseudo-Hermiticity in \cite{Mostafazadeh}. A Hamiltonian is said to be pseudo-Hermitian with respect to a positive-definite, Hermitian operator $\eta$ if it satisfies
\begin{equation}\label{eq:10}
\tilde{H}^{\dagger}  = \eta^{-1}\tilde{H}\eta.
\end{equation}
In the case of Hamiltonians with \PT-symmetry, the role of $\eta$ is played by \PP\CC. A convenient way to write the \CC\:operator was proposed in \cite{Bender_2006} 
\begin{equation}\label{eq:11}
\mathcal{C} = e^{Q}\mathcal{P} = \eta^{-1}\mathcal{P} ,
\end{equation}
where $Q$ is an antisymmetric Hermitian operator. Mostafazadeh \cite{Mostafazadeh2} has shown that the square root of the positive-definite Hermitian operator $\eta = e^{-Q}$ can be used to transform any non-Hermitian Hamiltonian with unbroken \PT-symmetry into a spectrally equivalent Hermitian Hamiltonian by means of a unitary``similarity transformation"\cite{PTvsDH}\cite{Jones_2005}. The transformation is as follows, the invertible operator $\rho = \sqrt{\eta}$ acts on the non-Hermitian \PT-symmetric Hamiltonian $\tilde{H}$ and returns an equivalent Hermitian Hamiltonian $\hat{H}$
\begin{equation}\label{eq:12}
\hat{H} = \rho^{-1}\tilde{H}\rho.
\end{equation}
Mostafazadeh \cite{Mostafazadeh}, conjectures that because \CPT-symmetry satisfies the postulates of quantum mechanics --whilst only ``swapping" the Hermiticity of a Hamiltonian by \CPT-symmetry-- then non Hermitian \CPT-symmetric theories are equivalent to certain non local Hermitian field theories. It is natural to notice that this feature could provide an advantage to simplify quantum mechanical calculations, as is explored in references \cite{Mostafazadeh},\cite{EquivalentHH},\cite{Pseudo-HermiticityIII},\cite{Jones_2005},\cite{taleof2potentials}.

%Interlude: 
\section{Two $2\times2$ Hamiltonians}\label{Inter}
A Hermitian Hamiltonian $\hat{H}$ and a \PT-symmetric non-Hermitian Hamiltonian $\tilde{H}$
\begin{align}\label{eq:13}
\hat{H} = \begin{pmatrix}
  &   \\
  &   \\
\end{pmatrix}, \quad\tilde{H} = \begin{pmatrix}
re^{i\theta} & s  \\
s & re^{-i\theta} \\
\end{pmatrix}
\end{align} 


\section{Bi-orthogonal systems}\label{BiS}

Suppose that a diagonalizable non-Hermitian Hamiltonian $\tilde{H}$ has a discrete spectrum and it commutes with the \PT\:operator.  Then $\tilde{H}$ has unbroken \PT-symmetry. There exists a Hilbert space $\mathcal{H}$ spanned by the eigenvectors of $\tilde{H}$ but because $\tilde{H}$ is non-Hermitian --with respect to a complete positive-definite inner product of $\mathcal{H}$-- the time evolution generated by $\tilde{H}$ will not be unitary\cite{Mostafazadeh}.
As explained in section \ref{CC}. For the assumed $\tilde{H}$ above we can construct a different complete positive-definite inner product by constructing and using the antisymmetric \CC\:operator that corresponds to $\tilde{H}$.
The assumed diagonalizability condition of $\tilde{H}$ may be viewed as a physical requirement without which an energy eigenbasis would not exist. To our knowledge all known non-Hermitian Hamiltonians that are used in physical applications are diagonalizable and therefore admit a complete bi-orthonormal set of eigenvectors. This set of bi-orthonormal eigenvectors is $\{(\ket{\psi_{n}, a}, \ket{\phi_{n}, a})\}$ and will satisfy the
following defining relations\cite{Pseudo-HermiticityIII}
\begin{align}
\tilde{H}\ket{\psi_{n}, a} = E_{n}\ket{\psi_{n}, a},\quad\tilde{H^{\dagger}}\ket{\phi_{n}, a} &= E^{*}_{n}\ket{\phi_{n}, a},\\
\nonumber\\
\braket{\phi_{m}, b|\psi_{n}, a} = \delta_{mn}\delta_{ab}&,\\
\nonumber\\
\sum^{}_{n}\sum_{a=1}^{d_n}\ket{\psi_{n}, a}\bra{\phi_{n}, a} = \hat{\mathbb{I}}&\\
\tilde{H} = \sum^{}_{n}\sum_{a=1}^{d_n}E_{n}\ket{\psi_{n}, a}\bra{\phi_{n}, a},\quad\tilde{H^{\dagger}} &= \sum^{}_{n}\sum_{a=1}^{d_n}E_{n}^{*}\ket{\phi_{n}, a}\bra{\psi_{n}, a}
\end{align}
where $n$ and $a$ are the spectral degeneracy levels, $d_n$ is the degree of degeneracy of $E_n$ and $\hat{\mathbb{I}}$ is the identity operator\cite{Pseudo-HermiticityIII}.

\chapter{Time evolution}\label{TEv}
Time-evolution in conventional quantum mechanics can be described using the unitary operator $\hat{U} = e^{-i\hat{H}t/\hbar}$.
$\hat{U}$ is unitary because the Hamiltonian $\hat{H}$ is Hermitian, that means that as the state $\vec{\psi}$ evolves in time, its norm remains constant in time. Hence time evolution of a state $\vec{\psi}$ from time $0 \rightarrow t$ is written as
\begin{equation}
\vec{\psi}(t) = \hat{U} \vec{\psi}(0).
\end{equation}
In Quantum mechanics, the norm of a state is interpreted as a probability, and this probability must remain constant in time. Probability growth or decay in time, means that the theory violates unitarity. As explained in section \ref{PT}, when the \PT-symmetry of $\hat{H}$ is unbroken we still have positive norm states. Furthermore, time evolution under the unbroken \PT-symmetric framework is indeed unitary\cite{Jones-Smith}\cite{ComplexExtension}\cite{Mostafazadeh2}.

\chapter{Consequences and applications}\label{Consq}

\chapter{Conclusion}\label{Conclusion}

\chapter{Appendix}\label{appendix}
\section*{One-to-one equivalence:\\ PT-symmetric and Hermitian quantum theories}

Exact \PT-symmetric quantum mechanics is understood to be 


% \PT\:symmetry is a dynamic property of a Hamiltonian $(\hat{H})$, in the sense that it is strongly dependent on the boundary conditions implemented on the eigenfunctions of $\hat{H}$.

% Let us consider the family of parametric Hamiltonians possessing \PT\:symmetry
% \begin{equation}\label{eq:4}
% \hat{H} = \hat{p}^2 + \hat{x}^{2}(i x)^{\epsilon} \quad\quad (\epsilon\:\mathrm{real}). 
% \end{equation}
% The Hamiltonian in \ref{eq:4} has unbroken \PT\:symmetry when $\epsilon \geq 0$ and broken \PT\:symmetry when $\epsilon < 0$. The energy spectrum of \ref{eq:4} has been proved to be real and bounded below in the region of unbroken \PT\:symmetry \cite{Dorey_2004}. 
% This particular family of Hamiltonians is useful to us as an example because we can demonstrate how by carefully choosing the boundary conditions associated with its eigenvalue problem  
% \begin{equation}\label{eq:5}
%  -\psi''_{n}(x) + x^2(ix)^{\epsilon}\psi_n(x) = E_n \psi_n(x), 
% \end{equation}
% where for $\epsilon\geq 0$ and for $x$ located in an infinite contour in the complex-x plane, $\psi(x) \rightarrow 0$ exponentially rapidly as $|x| \rightarrow \infty$. 

% There is evidence that suggests that, when properly normalised, the eigenfunctions $\psi(x)$ form a complete set\cite{MustaHbeHermitian}.


% \subsection{Stokes wedges and boundary conditions}\label{BCS}
% The class of \PT-symmetric Hamiltonians is larger than and includes Hermitian Hamiltonians because any real symmetric Hamiltonian is automatically \PT-symmetric\cite{MustaHbeHermitian}




\bibliographystyle{unsrturl_mod}
\bibliography{mybib}

\end{document}